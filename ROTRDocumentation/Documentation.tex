\documentclass{article}
\title{ROTR Experiment Documentation}
\author{Joe Collenette}
\date{}

\begin{document}
\maketitle

\section{Overview}
This document goes over how to setup and run the ROTR Experiment.
This assumes that you are using eclipse to run the document but should work with other IDE's.

There are some dependences that the projects uses:
\begin{description}
    \item[Prolog] you will to setup and install Prolog, and make sure that it can be run from the command line at any folder.
    \item[Simple Car Simulator] You will need to compile and export the SimpleCarSimulator to a jar file for the Experiment Files to run.
\end{description}

\section{Compiling Simple Car Simulator}
Firstly you should test that the SimpleCarSimulator works on your system.
This is a eclipse project, but should work on other systems.
There is a test experiment world file called ``ExampleWorldFile.txt", which you can use to make sure that the SimpleCarSimulator runs.

Using eclipse you can compile and run the project and should get a visual gui appear which allows you to load the ExampleWorldFile.
Once you are happy that the SimpleCarSimulator runs you need to export it to a JAR file.
In Eclipse you can go file->export which brings up the export screen.
From the export you should choose JAR file from the JAVA section.
Click Next. On the next screen the default settings should suffice, but you will need to give a location of where to save your JAR file.

\section{Running the experiments}
To run the experiments you will need to ensure that the exported JAR file from the SimpleCarSimulator is located in the bin folder.
I typically call the exported jar file ``SimpleCarSimulator.jar''.
You will also need to add this to your classpath when compiling to ensure that the project includes the jar file.

In eclipse you can right click on the ROTRExperiment project and choose the properties option. 
Then you navigate to the librarys tab, click on add external jar and choose the exported jar from the SimpleCarSimulator.

You should be able to compile and run the experiment application with the main method being in the RunExperiment.java file.
The experiment files are located in the simulated car folder which includes the (overtaking/trafficlight/turnright)experiment.txt
You can use the GUI to load this files and run each experiment.

\section{Writing your own world file}
The world file is a txt which defines what the simulation world is for an experiment.
The first two lines define the size of the world.
Width as number then the height of the world.
The next number should be the speed limit of the world, in number of cells per simulation step.
The following lines should define which cells can be driven in.
\begin{description}
    \item[$>$] Right only
    \item[$<$] Left only
    \item[\^] Up only
    \item[V] Down only
    \item[+] Any direction     
\end{description}

Following this defines where the cars start and where the traffic lights are and there white lines.
\begin{description}
    \item[trl x y sx sy] Defines a traffic light at point x y and the traffic should stop at sx sy
    \item[car name x y] Defines a car with a name, at point x y
    \item[car name x y (Info)] Defines a car with a name at point x y, and passed Info as a split list of strings. 
\end{description}
\end{document}